% Options for packages loaded elsewhere
\PassOptionsToPackage{unicode}{hyperref}
\PassOptionsToPackage{hyphens}{url}
%
\documentclass[
]{article}
\usepackage{amsmath,amssymb}
\usepackage{iftex}
\ifPDFTeX
  \usepackage[T1]{fontenc}
  \usepackage[utf8]{inputenc}
  \usepackage{textcomp} % provide euro and other symbols
\else % if luatex or xetex
  \usepackage{unicode-math} % this also loads fontspec
  \defaultfontfeatures{Scale=MatchLowercase}
  \defaultfontfeatures[\rmfamily]{Ligatures=TeX,Scale=1}
\fi
\usepackage{lmodern}
\ifPDFTeX\else
  % xetex/luatex font selection
\fi
% Use upquote if available, for straight quotes in verbatim environments
\IfFileExists{upquote.sty}{\usepackage{upquote}}{}
\IfFileExists{microtype.sty}{% use microtype if available
  \usepackage[]{microtype}
  \UseMicrotypeSet[protrusion]{basicmath} % disable protrusion for tt fonts
}{}
\makeatletter
\@ifundefined{KOMAClassName}{% if non-KOMA class
  \IfFileExists{parskip.sty}{%
    \usepackage{parskip}
  }{% else
    \setlength{\parindent}{0pt}
    \setlength{\parskip}{6pt plus 2pt minus 1pt}}
}{% if KOMA class
  \KOMAoptions{parskip=half}}
\makeatother
\usepackage{xcolor}
\usepackage[margin=1in]{geometry}
\usepackage{graphicx}
\makeatletter
\def\maxwidth{\ifdim\Gin@nat@width>\linewidth\linewidth\else\Gin@nat@width\fi}
\def\maxheight{\ifdim\Gin@nat@height>\textheight\textheight\else\Gin@nat@height\fi}
\makeatother
% Scale images if necessary, so that they will not overflow the page
% margins by default, and it is still possible to overwrite the defaults
% using explicit options in \includegraphics[width, height, ...]{}
\setkeys{Gin}{width=\maxwidth,height=\maxheight,keepaspectratio}
% Set default figure placement to htbp
\makeatletter
\def\fps@figure{htbp}
\makeatother
\setlength{\emergencystretch}{3em} % prevent overfull lines
\providecommand{\tightlist}{%
  \setlength{\itemsep}{0pt}\setlength{\parskip}{0pt}}
\setcounter{secnumdepth}{-\maxdimen} % remove section numbering
\ifLuaTeX
  \usepackage{selnolig}  % disable illegal ligatures
\fi
\usepackage{bookmark}
\IfFileExists{xurl.sty}{\usepackage{xurl}}{} % add URL line breaks if available
\urlstyle{same}
\hypersetup{
  pdftitle={Análisis de la Distribución y Diversidad de Prestadores de Servicios de Salud en Colombia},
  hidelinks,
  pdfcreator={LaTeX via pandoc}}

\title{Análisis de la Distribución y Diversidad de Prestadores de
Servicios de Salud en Colombia}
\author{}
\date{\vspace{-2.5em}}

\begin{document}
\maketitle

\begin{verbatim}
## [1] 0
\end{verbatim}

\begin{verbatim}
## [1] 0
\end{verbatim}

\subsubsection{Distribución de Prestadores por Departamento y
Municipio}\label{distribuciuxf3n-de-prestadores-por-departamento-y-municipio}

\begin{center}\includegraphics{PruebaTecnicaAnalistaDatos_files/figure-latex/unnamed-chunk-7-1} \end{center}

La concentración de prestadores de servicios de salud por departamento
en Colombia revela patrones claros que reflejan la distribución desigual
de servicios de salud a lo largo del país. Los departamentos Antioquia,
Valle del Cauca, Atlántico, Santander y el Distrito Capital muestran una
alta concentración de prestadores de servicios de salud. Esta alta
concentración se puede atribuir principalmente a que estas regiones
albergan grandes centros urbanos con poblaciones densas y una alta
demanda de servicios de salud. La presencia de numerosos hospitales,
clínicas y otros proveedores responde a la necesidad de atención médica
de una población numerosa y diversa. En contraste, los departamentos
periféricos como Amazonas, Guainía y Vichada muestran una presencia
mucho más tenue de prestadores de servicios de salud. La falta de
presencia en estas zonas periféricas puede atribuirse a su baja densidad
poblacional, condiciones geográficas difíciles de acceso y la escasez de
infraestructura adecuada para el establecimiento de servicios de salud.

\subsubsection{Análisis de Correlación entre Población y Prestadores de
Servicios de
Salud}\label{anuxe1lisis-de-correlaciuxf3n-entre-poblaciuxf3n-y-prestadores-de-servicios-de-salud}

\begin{center}\includegraphics{PruebaTecnicaAnalistaDatos_files/figure-latex/unnamed-chunk-9-1} \end{center}

La correlación de 0.979 sugiere que hay una asociación fuerte y positiva
entre la población de un municipio y la cantidad de prestadores de
servicios de salud que se encuentran en ese municipio. Esto significa
que, en general, a medida que aumenta la población de un municipio,
tiende a haber más prestadores de servicios de salud disponibles. Esto
es coherente con la lógica de oferta y demanda en servicios de salud,
donde áreas más pobladas pueden soportar una mayor cantidad de
proveedores de servicios de salud debido a la demanda potencialmente más
alta.

\newline

\subsubsection{Análisis de Regresión entre Población, Superficie y
Prestadores de Servicios de
Salud}\label{anuxe1lisis-de-regresiuxf3n-entre-poblaciuxf3n-superficie-y-prestadores-de-servicios-de-salud}

\begin{verbatim}
## 
## \begin{table}[!htbp] \centering 
##   \caption{Modelo de Regresión: Número de Prestadores vs. Población/Superficie} 
##   \label{} 
## \small 
## \begin{tabular}{@{\extracolsep{5pt}}lcc} 
## \\[-1.8ex]\hline 
## \hline \\[-1.8ex] 
##  & \multicolumn{2}{c}{\textit{Dependent variable:}} \\ 
## \cline{2-3} 
## \\[-1.8ex] & \multicolumn{2}{c}{num\_prestadores} \\ 
## \\[-1.8ex] & (1) & (2)\\ 
## \hline \\[-1.8ex] 
##  poblacion & 0.002$^{***}$ & 0.002$^{***}$ \\ 
##   & (0.00001) & (0.00001) \\ 
##   & & \\ 
##  Superficie &  & $-$0.003$^{**}$ \\ 
##   &  & (0.001) \\ 
##   & & \\ 
##  Constant & $-$34.779$^{***}$ & $-$32.063$^{***}$ \\ 
##   & (3.620) & (3.762) \\ 
##   & & \\ 
## \hline \\[-1.8ex] 
## Observations & 986 & 986 \\ 
## R$^{2}$ & 0.959 & 0.960 \\ 
## Adjusted R$^{2}$ & 0.959 & 0.959 \\ 
## Residual Std. Error & 111.906 (df = 984) & 111.590 (df = 983) \\ 
## F Statistic & 23,190.430$^{***}$ (df = 1; 984) & 11,664.140$^{***}$ (df = 2; 983) \\ 
## \hline 
## \hline \\[-1.8ex] 
## \textit{Note:}  & \multicolumn{2}{r}{$^{*}$p$<$0.1; $^{**}$p$<$0.05; $^{***}$p$<$0.01} \\ 
## \end{tabular} 
## \end{table}
\end{verbatim}

Los modelos sugieren que la población de un municipio es un predictor
significativo del número de prestadores de servicios de salud. El
coeficiente de 0.002 nos muestra que un aumento unitario en la población
de un municipio se asocia, en promedio, con un aumento de 0.002 en el
número de prestadores de servicios de salud.

Adicionalmente se agrega la variable de superficie para analizar si la
extensión territorial de un municipio tiene impacto en la disponibilidad
de prestadores de servicios de salud. El coeficiente (-0.003) indica que
a medida que aumenta la superficie de un municipio, se espera una
disminución en el número de prestadores de salud.

Esto podría indicar que la asignación de recursos y la accesibilidad
pueden ser más complejas en municipios más extensos geográficamente.
Esto podría deberse a desafíos logísticos y de infraestructura que
pueden hacer menos atractivo para los prestadores de salud establecerse
en áreas con mayores extensiones territoriales.

\subsubsection{Análisis de Diversidad de
Prestadores}\label{anuxe1lisis-de-diversidad-de-prestadores}

\begin{center}\includegraphics{PruebaTecnicaAnalistaDatos_files/figure-latex/unnamed-chunk-11-1} \end{center}

Regiones Centrales, el Altiplano Cundiboyacense y Antioquia tienen altos
índices de diversidad. Esto puede deberse a una infraestructura de salud
más desarrollada y una población más grande y diversa, lo que permite
una mayor variedad de prestadores de servicios de salud.

Por otro lado, las regiones periféricas y menos pobladas, como Amazonas,
Guainía y Vaupés, muestran bajos índices de diversidad. Estos son
típicamente departamentos con menor densidad poblacional y menor
desarrollo de infraestructura de salud. en este tipo de regiones las
dificultades logísticas y la falta de infraestructura pueden limitar la
presencia de diversos tipos de prestadores, haciendo que los prestadores
pueden encontrar poco atractivo establecerse en estas áreas debido a los
altos costos operativos y logísticos.

\subsubsection{Concluciones}\label{concluciones}

El análisis de los prestadores de servicios de salud en Colombia revela
una distribución desigual, donde regiones urbanas densamente pobladas
como Bogotá, Antioquia y Valle del Cauca exhiben una alta concentración
y diversidad de prestadores, reflejando una respuesta adaptativa a una
demanda variada de servicios de salud. En contraste, áreas periféricas
como Amazonas y Guainía muestran una presencia limitada de prestadores,
atribuible a desafíos logísticos y falta de infraestructura. La
correlación positiva entre la población y el número de prestadores
subraya la importancia de políticas que fomenten la inclusión y
diversificación de prestadores, especialmente en áreas menos
desarrolladas geográficamente. Estos hallazgos destacan la necesidad de
políticas regionales que optimicen la coordinación y calidad de los
servicios de salud para mejorar el acceso equitativo a nivel nacional.

\end{document}
